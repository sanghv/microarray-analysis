\documentclass{beamer}

% Theme
\usetheme{Hannover}
\usecolortheme{crane}
\definecolor{gray10percent}{RGB}{229, 229, 229}
\definecolor{blockgray}{RGB}{198, 188, 169}

% Packages
\usepackage{amsmath}
\usepackage{graphicx}
\usepackage{rotating}
\usepackage{etoolbox}
\usepackage{xfrac}
\usepackage{array}

\usepackage{color}
\usepackage{listings}
\lstset
    {
       language=C,
      captionpos=b,
      tabsize=3,
      keywordstyle=\color{blue},
      commentstyle=\color[rgb]{0.133,0.545,0.133},
      stringstyle=\color{red},
      breaklines=true,
      showstringspaces=false,
      basicstyle=\footnotesize,
      emph={label},
      keepspaces=true,
      emph={H,G,X,K,KPC},
      emphstyle={\bfseries}
    }
    \newcommand{\codepause}{\pause \vspace{-0.165in}}
    
    \newcommand\ConstrainedBox[3]{
      \makebox{\parbox[t][#1][c]{#2}{\centering#3}}}
    
    % Decide if notes are shown
    \newtoggle{useNotes}
    %\toggletrue{useNotes}
    \togglefalse{useNotes}

    \iftoggle{useNotes}
             {
               \usepackage{pgfpages}
               \setbeameroption{show notes on second screen}
             }
    

      
             % Define title page
             \title
                 {
                   KPCA-Biplots
                 }
                 \subtitle{An implementation and further exploration}
                 \author
                     {
                       Michael Semeniuk, Albert Steppi, and \linebreak
                       Christopher Wolas
                     }
                     \date
                         {
                           \begin{block}{}
                             \begin{itemize}
                             \item[~]
                               {
                                 \underline{Mining Gene Expression Profiles:
                                   An Integrated} \linebreak
                                 \underline{Implementation of Kernel Principal
                                   Component} \linebreak
                                 \underline{Analysis and Singular Value
                                   Decomposition}
                                 \begin{itemize}
                                   \item[~] Reverter et al, Genomics Proteomics
                                     Bioinformatics (2010)
                                 \end{itemize}
                               }
                             \end{itemize}
                           \end{block}
                         }

                         \begin{document}


                         \begin{frame}
                           \titlepage
                         \end{frame}

                         \begin{frame}[t]
                           \frametitle{Singular Value Decomposition}
                           A means of decomposing a matrix into simpler 
                           components that reflect the overall structure.
                           \begin{equation}                             
                             X\;=\;UDV^{T}                             
                           \end{equation}
                           When $X$ is a square matrix, the geometry is simple.
                           $U$ and $V^{T}$ are rotations and $D$
                           is a scaling. So it says the transformation $X$ can
                           be accomplished by first rotating, then scaling, then
                           rotating again.

                           \begin{block}{What this does for us.}
                             \begin{itemize}
                               \item Finds a natural system of coordinates for
                                 the data.
                               \item The coordinates are orthogonal, and are 
                                 ordered by how much variance in the data they
                                 account for.
                               \item Allows you to reduce the dimensionality of
                                 your problem. The first couple of coordinates
                                 account for most of the variance in the data.
                             \end{itemize}
                           \end{block}                           
                         \end{frame}
                         
                         \begin{frame}[t]
                           \frametitle{Kernel Principal Component Analysis}
                           \begin{itemize}
                           \item
                             Finding a natural set of coordinates for the data
                             that are ordered by how much variance in the data
                             they account for is known as principal component
                             analysis. 
                           \item
                             This allows us to reduce dimensionality and thus
                             visualize the data, but the relations between
                             gene expressions and tissue traits are often
                             highly nonlinear.
                           \end{itemize}
                           \begin{block}{Working with nonlinearity}
                             \begin{itemize}
                             \item
                               As with support vector machines, mapping to a
                               higher dimensional space with the help of the
                               kernel trick can help when dealing with 
                               nonlinearity.
                             \item
                               In kernel principal component analysis you map
                               to a higher dimensional space, do principal 
                               component analysis, then you can project your
                               data onto the first couple of principal 
                               components.
                             \end{itemize}                             
                           \end{block}
                         \end{frame}
                         \begin{frame}[t]
                           \frametitle{KPCA Biplot}
                           Using singular value decomposition, we can factor
                           are data matrix as follows
                           \begin{displaymath}
                             X\;=\;GH^{T}
                           \end{displaymath}
                           
                           \begin{itemize}
                           \item
                             Here $G$ contains information about the microarray
                             samples and $H$ contains information about the genes
                             under investigation.
                           \item
                             To create a gene expression biplot you project
                             both $G$ and $H$ onto the first two principal
                             components and map them together in one plot.
                           \item
                             To create a KPCA biplot you do kernel principal
                             component analysis on $H$, map $G$ into the kernel
                             space as well, then project both of them onto the 
                             first two principal components from KPCA and plot 
                             them together.
                           \item
                             This seems kind of dodgy. You do one form of PCA
                             when you do the singular value decomposition, then
                             you do a round of KPCA. This might not be the best
                             approach.
                           \end{itemize}
                         \end{frame}
                           
                           



                         \end{document}
